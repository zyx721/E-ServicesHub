\documentclass[12pt,a4paper]{report}
\usepackage[utf8]{inputenc}
\usepackage[francais]{babel}
\usepackage[T1]{fontenc}
\usepackage{amsmath}
\usepackage{amsfonts}
\usepackage{amssymb}
\usepackage{graphicx}
\usepackage{float}
\usepackage{tabularx}
\usepackage{array}
\usepackage{rotating}
\usepackage{fancyhdr}
\usepackage{lmodern}
\usepackage{algorithm}
\usepackage{algpseudocode}
\usepackage{float}
\usepackage{geometry}
\usepackage[Glenn]{ fncychap }
\usepackage{multirow}
\usepackage{setspace}
\pagestyle{fancy}
\lhead{\leftmark}
\rhead{}
\geometry{left=2.5cm, right=2cm, top=2.5cm, bottom=2.5cm}
\onehalfspacing

\begin{document}
\thispagestyle{empty}

\begin{center}
ALGERIAN DEMOCRATIC AND PEOPLE'S REPUBLIC\\ Ministry of Higher Education and Scientific Research\\
\textbf{Numedia Institute of Technologie}\\
\end{center}
\begin{figure}[H]
  \centering
  \includegraphics[width=0.2\linewidth]{images.jpg}
\end{figure}
\begin{center}
\begin{LARGE}
\textbf{E-Services platform}\\
\end{LARGE}
     
      
\begin{large}Presented for connect users with reliable service providers, simplifying access to essential physical services while enhancing convenience and trust\\
\textbf{                      }\\
\textbf{                      }\\
IN : \textbf{Software Engeneering}\\
speciality: \textbf{Artificial Intelligence}\\
By : \textbf{AL Asmar Anas,Hechiche Nouha }\\
\textbf{Bel mati zyad ,Naji Fares,Zoubir}\\


\end{large}
\end{center}
\vfill

\fbox{\begin{minipage}[c]{0.9\linewidth}
\begin{center}
\begin{large}
\textbf{HelpHub platform}
\end{large}
\end{center}
\end{minipage}}
\setcounter{page}{0}
\newpage
\pagenumbering{Roman}

\newpage
\tableofcontents
\newpage

\chapter{Overview}
\section{Introduction}
Our E-Services Platform is a digital solution designed to bridge the gap between users seeking physical services (like plumbing, housekeeping, and other home maintenance tasks) and service providers who offer these services. By leveraging technology, the platform simplifies the process of finding and hiring help, making it more efficient and user-friendly.
\section{Purpose and Objectives}
1.Connecting Users and Providers: The primary objective of the platform is to create a seamless connection between individuals who need services and professionals who can provide them. This eliminates the traditional barriers associated with service hiring, such as time-consuming searches and unreliable referrals.

2.Enhancing Accessibility: The platform aims to make essential services accessible to a broader audience. Whether someone lives in an urban area or a rural community, the platform provides a way to find nearby service providers, ensuring that everyone can access help when needed.

3.Streamlining Service Requests: By allowing users to post specific service requests, the platform reduces the hassle of searching through numerous listings. Users can describe their needs in detail and receive responses from providers who can meet those needs, thus speeding up the process
\section{Market Context}
1.Growing Demand for E-Services: As society increasingly shifts towards digital solutions, there is a growing demand for online platforms that can facilitate everyday tasks. The rise of the gig economy further emphasizes the need for reliable connections between consumers and service providers.

2.Competitive Landscape: The market includes several established players like TaskRabbit, Thumbtack, and HomeAdvisor, each with its strengths and weaknesses. Understanding this competitive landscape allows the E-Services Platform to position itself effectively by addressing gaps in the current offerings.
\section{User-Centric Approach}
1.Focus on User Experience: The platform is designed with a user-first mentality, ensuring that navigation, service request submissions, and communication with providers are straightforward and intuitive. This focus on user experience is essential for attracting and retaining users.

\chapter{Our Competitors}
\section{TaskRabbit}
\end{document}