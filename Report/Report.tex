\documentclass[12pt,a4paper]{report}
\usepackage[utf8]{inputenc}
\usepackage[francais]{babel}
\usepackage[T1]{fontenc}
\usepackage{amsmath}
\usepackage{amsfonts}
\usepackage{amssymb}
\usepackage{graphicx}
\usepackage{float}
\usepackage{tabularx}
\usepackage{array}
\usepackage{rotating}
\usepackage{fancyhdr}
\usepackage{lmodern}
\usepackage{algorithm}
\usepackage{algpseudocode}
\usepackage{float}
\usepackage{geometry}
\usepackage[Glenn]{ fncychap }
\usepackage{multirow}
\usepackage{setspace}
\pagestyle{fancy}
\lhead{\leftmark}
\rhead{}
\geometry{left=2.5cm, right=2cm, top=2.5cm, bottom=2.5cm}
\onehalfspacing

\begin{document}
\thispagestyle{empty}

\begin{center}
ALGERIAN DEMOCRATIC AND PEOPLE'S REPUBLIC\\ Ministry of Higher Education and Scientific Research\\
\textbf{Numedia Institute of Technologie}\\
\end{center}
\begin{figure}[H]
  \centering
  \includegraphics[width=0.4\linewidth]{images.png}
\end{figure}
\begin{center}
\begin{LARGE}
\textbf{E-Services platform}\\
\end{LARGE}
     
      
\begin{large}Presented for connect users with reliable service providers, simplifying access to essential physical services while enhancing convenience and trust\\
\textbf{                      }\\
\textbf{                      }\\
\textbf{                      }\\
IN : \textbf{Software Engeneering}\\
speciality: \textbf{Artificial Intelligence}\\
By : \textbf{Alasmar Anas,Abdi Adem Fares }\\
\textbf{Benmati ziad ,Nouha Hechiche, Zobir Abd Raouf}\\


\end{large}
\end{center}
\vfill

\fbox{\begin{minipage}[c]{0.9\linewidth}
\begin{center}
\begin{large}
\textbf{HANINI}
\end{large}
\end{center}
\end{minipage}}
\setcounter{page}{0}
\newpage
\pagenumbering{Roman}

\newpage
\tableofcontents
\newpage

\chapter{Overview}
\section{Introduction}
Our E-Services Platform is a digital solution designed to bridge the gap between users seeking physical services (like plumbing, housekeeping, and other home maintenance tasks) and service providers who offer these services. By leveraging technology, the platform simplifies the process of finding and hiring help, making it more efficient and user-friendly.
\section{Purpose and Objectives}
1.Connecting Users and Providers: The primary objective of the platform is to create a seamless connection between individuals who need services and professionals who can provide them. This eliminates the traditional barriers associated with service hiring, such as time-consuming searches and unreliable referrals.

2.Enhancing Accessibility: The platform aims to make essential services accessible to a broader audience. Whether someone lives in an urban area or a rural community, the platform provides a way to find nearby service providers, ensuring that everyone can access help when needed.

3.Streamlining Service Requests: By allowing users to post specific service requests, the platform reduces the hassle of searching through numerous listings. Users can describe their needs in detail and receive responses from providers who can meet those needs, thus speeding up the process
\section{Market Context}
1.Growing Demand for E-Services: As society increasingly shifts towards digital solutions, there is a growing demand for online platforms that can facilitate everyday tasks. The rise of the gig economy further emphasizes the need for reliable connections between consumers and service providers.

2.Competitive Landscape: The market includes several established players like TaskRabbit, Thumbtack, and HomeAdvisor, each with its strengths and weaknesses. Understanding this competitive landscape allows the E-Services Platform to position itself effectively by addressing gaps in the current offerings.
\textbf{                      }\\
\section{User-Centric Approach}
1.Focus on User Experience: The platform is designed with a user-first mentality, ensuring that navigation, service request submissions, and communication with providers are straightforward and intuitive. This focus on user experience is essential for attracting and retaining users.

\chapter{Our Competitors}
\section{TaskRabbit}
\begin{figure}[ht]
    \begin{center}
        \includegraphics[scale=0.8]{taskrabit.png}
         \includegraphics[scale=0.8]{task2.png}
     \end{center}
\end{figure}
\subsection*{Definition}
TaskRabbit is an online marketplace that connects users with local freelancers who offer a variety of services, such as home repairs, cleaning, and personal assistance. It allows clients to easily hire help for everyday tasks, promoting a flexible gig economy.
\subsection*{Disadvantages vs. HANINI Advantages}
\underline {Disadvantage}: High service fees can deter cost-sensitive customers.\\
\textbf{                      }\\
\textbf{                      }\\
\underline {HANINI Advantage}: HANINI will implement a more transparent and lower fee structure, making services more accessible for budget-conscious users.



\section{Thumbtackt}
\begin{figure}[ht]
    \begin{center}
        \includegraphics[scale=0.7]{Thum.png}
         
     \end{center}
\end{figure}
\subsection*{Definition}
Thumbtack is an online platform that connects users with local service professionals for various needs, such as home improvement, events, and personal services. Users can request quotes, compare professionals, and hire based on reviews and pricing, streamlining the search for reliable help.
\subsection*{Disadvantages vs. HANINI Advantages}
\underline {Disadvantage}: The complex and multi-step process for requesting services can confuse users, leading to potential drop-offs.\\
\textbf{                      }\\
\textbf{                      }\\
\underline {HANINI Advantage}: HANINI will offer a streamlined, user-friendly interface that simplifies the booking process, ensuring a seamless experience from search to booking.


\section{HomeAdvisort}
\begin{figure}[ht]
    \begin{center}
        \includegraphics[scale=0.9]{home.jpg}
         
     \end{center}
\end{figure}
\subsection*{Definition}
HomeAdvisor is an online service that connects homeowners with local contractors and service professionals for home improvement projects and repairs. It offers tools for finding, comparing, and reviewing service providers, making it easier for users to hire reliable help for their needs.
\subsection*{Disadvantages vs. HANINI Advantages}
\underline {Disadvantage}:Aggressive marketing tactics can create distrust among users, and higher consumer costs may arise from provider fees.\\
\textbf{                      }\\
\textbf{                      }\\
\underline {HANINI Advantage}:  HANINI will focus on building trust through genuine user reviews and transparent pricing, fostering a reliable environment for both clients and service providers.


\section{Fiverrt}
\begin{figure}[ht]
    \begin{center}
        \includegraphics[scale=0.8]{fir.jpg}
         
     \end{center}
\end{figure}
\subsection*{Definition}
Fiverr is an online platform that connects freelancers with clients seeking various services, including design, writing, and marketing. Users can easily browse and hire professionals for projects starting at just 5 dolars, streamlining the hiring process.
\subsection*{Disadvantages vs. HANINI Advantages}
\underline {Disadvantage}: Quality control issues mean users might face inconsistent service quality and an overwhelming number of options to sift through.\\
\textbf{                      }\\
\textbf{                      }\\
\underline {HANINI Advantage}:  With AI-based identity verification and a robust review system, HANINI ensures that only verified, high-quality service providers are listed, making it easier for users to find reliable help.


\section{Angie’s List}
\begin{figure}[ht]
    \begin{center}
        \includegraphics[scale=0.5]{ang1.png}
         \includegraphics[scale=0.5]{ang2.png}
     \end{center}
\end{figure}
\subsection*{Definition}
Angie’s List is an online directory that connects users with local service providers for home improvement, repairs, and maintenance. It features user reviews and ratings to help consumers find trustworthy professionals for their projects.
\subsection*{Disadvantages vs. HANINI Advantages}
\underline {Disadvantage}: Membership fees can be a barrier for potential users, and the dated interface may detract from user engagement\\
\textbf{                      }\\
\textbf{                      }\\
\underline {HANINI Advantage}:  HANINI will be free to use for clients, removing financial barriers, and will feature a modern, intuitive interface that enhances user engagement and satisfaction.
\chapter{Unique Additions of the HANINI Platform}
\section{AI-Based Identity Verification}
\underline {Advantage}: Enhances trust and security by verifying service providers through their national ID and live photo comparison, reducing the risk of fraud and ensuring reliable services.

\section{Localized Service Focus}
\underline {Advantage}: Tailored specifically for the Algerian market, HANINI addresses local needs and preferences, making it easier for users to find relevant services in their area.

\section{Streamlined User Experience}
\underline {Advantage}:  A user-friendly interface simplifies the booking process, making it easy for clients to search, book, and manage services, which reduces drop-off rates and enhances customer satisfaction.


\section{Premium Subscription Options}
\underline {Advantage}:  Service providers can opt for premium subscriptions to boost their visibility, which not only benefits them but also allows clients to discover higher-quality services more easily.

\section{Integration with Local Payment Gateways}
\underline {Advantage}:  Seamless integration with popular local payment methods makes it convenient for users to manage transactions related to premium subscriptions, ensuring a smooth financial experience.
\chapter{Conclusion}
In contrast to existing competitors, HANINI offers a compelling alternative by addressing their key shortcomings. By providing lower costs, a user-friendly experience, verified service providers, and a trustworthy environment, HANINI positions itself as a superior choice for users seeking local services in Algeria. This strategic focus on user needs will allow HANINI to effectively capture and retain its target audience.
\end{document}